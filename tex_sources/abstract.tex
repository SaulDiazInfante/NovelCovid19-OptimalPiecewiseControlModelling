%!TEX root = main.tex
\begin{abstract}
<<<<<<< HEAD
We formulate a controlled system of ordinary differential equations, with
vaccination and lockdown interventions as controls, to simulate the mitigation
of COVID-19. The performance of the controls is measured through a cost
functional involving vaccination and lockdown costs as well as the burden of
COVID19 quantified in DALYs. We calibrate parameters with data from Mexico City
and Valle de Mexico.  By using differential evolution, we minimize the cost
functional subject to the controlled system and find optimal policies that are
constant in time intervals of a given size. The main advantage of these
policies relies on its practical implementation since the health authority has
to make only a finite number of different decisions. Our methodology to find
optimal policies is relatively general, allowing changes in the dynamics, the
=======
    We formulate a controlled system of ordinary differential equations, with 
vaccination and lockdown interventions as controls, to simulate the mitigation 
of COVID-19. The performance of the controls is measured through a cost 
functional involving vaccination and lockdown costs as well as the burden of 
COVID19 quantified in DALYs. We calibrate parameters with data from Mexico City 
and Valle de Mexico.  By using differential evolution, we minimize the cost 
functional subject to the controlled system and find optimal policies that are 
constant in time intervals of a given size. The main advantage of these 
policies relies on its practical implementation since the health authority has 
to make only a finite number of different decisions. Our methodology to find 
optimal policies is relatively general, allowing changes in the dynamics, the 
>>>>>>> 556be0e4dd15d8351e184ff4da1790e9f26cd928
cost functional, or the frequency the policymaker changes actions.
\end{abstract}
