%!TEX root = main.tex
At the date of writing this manuscript, the USA is running its COVID-19 
vaccination with Pfizer-BioNTech vaccine. This vaccine development along with 
Astra-Zeneca,Cansino, Sputnik V, Novavax among others' promise to deliver 
enough dosesfor Latin America. In Mexico, particularly, the first stock with 
around \num{40 000} shots has arrived past Christmas. In past October, WHO 
established a recommended protocol for prioritizing access to this
pharmaceutical hope giving clear lines about who has to be vaccinated first
and why. However, each developed vaccine implies different issues around its
application. For example, Pfizer-BioNTech vaccine requires two doses and
particular logistics requirements that demand special services. In
Mexico, despite Pfizer-BioNTech has been taking the responsibility to 
capacitate personnel that manage the vaccination, there is an explicit demand 
for logistics resources that limit the institutions' response. On the hand, 
nonpharmaceutical interventions (NPIs), like a lockdown, also involve economic 
costs.  Our research in this manuscript explores the effect of two 
interventions, vaccination and lockdown, to mitigate the  propagation of 
COVID-19.


Among the related literature about the two interventions we deal with in this 
paper, we can mention the following. The problem of who is vaccinated first, 
when the number of  available shots is limited, has been transformed into an 
optimal allocation problem of vaccine doses in \cite{Bubar2020,Matrajt2020}.  
These articles give answers to the critical question of how much doses 
allocate to each different group according to risk and age to minimize the 
burden of COVID-19. In our study, we take the allocation for granted and 
consider only the vaccination rate. 
Further, papers modeling NPIs consider the diminish of contact rates---by
reducing mobility---or modulating parameters regarding the generation of new
infections by linear controls \cite{Naraigh2020,Ullah2020},
Lockdown--Quarantine \cite{Mandal2020},  shield immunity
\cite{Weitz2020}. In addition, Libotte et. al. reports in \cite{Libotte2020} 
optimal vaccination strategies for COVID-19. 

    Since health services' response will be limited by the vaccine stock
and logistics costs, implementing in parallel NPIs is imminent. We focus on
formulating and studying via simulation a Lockdown-Vaccination system by 
consider the vaccine recently approved by  Mexico Health Council.     We aim to 
design a schedule for dose application subject to a given vaccine stock that 
will be applied in a given period of time. For this purpose, we formulate an 
optimal control problem that minimizes the burden of  COVID-19 in DALYs 
\cite{WhoDALY}, the cost generated by   running the vaccination 
campaign, and economic damages due to lockdown.

    One of the main features of our model is that we consider piecewise
constant control policies instead of general measurable control policies
\textemdash also called permanent controls\textemdash to minimize a cost 
functional. General control policies are difficult to implement since the 
authority has to make different choices every permanently. The optimal policies 
we find are constant in each interval of time and hence these policies are 
easier to implement. To the best of our knowledge, our manuscript is the first 
optimal control model with both lockdown and vaccination strategies that are 
easy to implement in the sense described above. 


In \Cref{sec:Covid19_spread}, we formulate the
basic spread model for COVID19 and calibrate its parameters. Then,
\Cref{sec:vaccination_model} establishes the lockdown--vaccination model and
discusses the regarding reproductive number in
\Cref{sec:reproductive_number}. 
In \Cref{sec:optimal_controlled} we describe the optimal control problem 
which consists in minimizing a cost functional subject to controlled 
lockdown--vaccination system. The optimal policies we find, by solving 
numerically the optimal control problem, are presented in 
\Cref{sec:numerical_experiments}. We conclude with some final comments in 
\Cref{sec:conclusion}.