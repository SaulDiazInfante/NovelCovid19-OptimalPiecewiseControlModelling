  %!TEX root = main.tex
\paragraph{Background}
 At the date of writing this manuscript, a Pfizer-Biotech vaccine is
implementing in the USA. This vaccine development among Astra-Zeneca, Cansino,
Sputnik, Novavax another's promises deliver sufficient doses for
Latinoamerica, particularly in Mexico this past Christmas has been arriving
the firs stock with around 40 000 amounts. In October, WHO established a
recommended protocol for prioritizing access to this pharmaceutical hope,
given clear lines about who has to be vaccinated first and why.
However, each vaccine development implies different issues to its application.
For example, the Pfizer-Biotech vaccine requires two doses and very
particularly logistic requirements that demand special services. In Mexico,
despite Pfizer taking the responsibility to capacitate and help manage the
immunization, we observe an explicit demand for health-logistic resources that
limit our institutions' response. Thus our research interest in this manuscript
explores the effect of the combined interventions Lockdown-Vaccination to
mitigate COVID-19.


\paragraph{Litterature review}
    The issue of how vaccine first has been traduced as an optimal allocation
problem of vaccine doses, we recommend to the interested reader the
articles Bubar(2020) and  Matrajt(2020).  These articles consider scenarios
where the health services response and vaccine stock achieve the given
vaccination policy's objectives and respond to the critical question of how
much doses allocate to each different group according to risk and age to
minimize the burden of COVID-19.

    Early articles about COVID-19 optimal intervention modeling mainly focus on
Nonpharmaceutical interventions (NPIs). Mostly these works understand the
control strategy as the diminish of contact rates by reducing mobility or
modulating parameters regarding the generation of new infections by linear
controls (see for example Naraigh(2020),  Ullah(2020)), Lockdown-Quarantine
Manda(l2020),  shield immunity Weitz(2020).

    Libotte et. al. reports in (Libotte(2020) an Optimal vaccination strategies
for COVID19.
\subsection{Contribution and main objectives}
    Our manuscript is the first contribution modeling with optimal control of
    Lockdown-Vaccination strategies' effect to the best of our knowledge. Since
    health services' response will be limited by the vaccine stock, and
    logistics to implement in parallel NPIs is mandatory. We focus on
    formulating and studying via simulation the system Lockdown-Vaccination
    with recent and approved vaccine profile by the  Mexico Health council and
    developing optimal policies for the Lockdown release-input and Vaccine
    application doses.
\paragraph{Vaccine development}
    According to the Gouvernamental comunicated in Dec. Mexico treated
    \SI{36000000}{doses} Pfizer-Biontech, \SI{76000000}{doses} with Aztra
    Seneca  \SI{18000000}{doses} of Cansino-BIO

\paragraph{Problem setup}

\paragraph{Piecewise optimal policies}
    Comment about the solution of the underlying Optimal Control Problem
    \comment[id=SDIV]{David}

One of the main features of our model is that we consider piecewise constant
control policies instead of general measurable control policies. General
control policies are difficult to implement since the authority has to make
different choices every instant. The optimal policies we find are constant in
each interval of time and hence these policies are easier to implement.

Optimal control problems with piecewise constant policies have been widely studied:  solution method \cite{MR3228405}, convergence \cite{MR3627992}.

However, to the best of our knowledge, this is the first application of such
policies in epidemics Vaccination-Lockdown control for COVID-19.


\paragraph{Papaer structure}