  %!TEX root = main.tex
\paragraph{Background}
         At the date of writing this manuscript, a Pfizer-Biotech vaccine is
    implementing in the USA. This vaccine development among Astra-Zeneca,
    Cansino, Sputnik, Novavax another's promises deliver sufficient doses
    for Latinoamerica, particularly in Mexico this past Christmas has been
    arriving the firs stock with around 40 000 amounts. In October, WHO
    established a recommended protocol for prioritizing access to this
    pharmaceutical hope, given clear lines about who has to be vaccinated first
    and why.However, each vaccine development implies different issues to its
    application. For example, the Pfizer-Biotech vaccine requires two doses and
    very particularly logistic requirements that demand special services. In
    Mexico, despite Pfizer taking the responsibility to capacitate and help
    manage the immunization, we observe an explicit demand for health-logistic
    resources that limit our institutions' response. Thus our research interest
    in this manuscript explores the effect of the combined interventions
    Lockdown-Vaccination to mitigate COVID-19.


\paragraph{Litterature review}
        The issue of how vaccine first has been traduced as an optimal
    allocation problem of vaccine doses, we recommend to the interested reader
    the articles \cite{Bubar2020,Matrajt2020}.  These articles
    consider scenarios where the health services response and vaccine stock
    achieve the given vaccination policy's objectives and respond to the
    critical question of how much doses allocate to each different group
    according to risk and age to minimize the burden of COVID-19.

        Early articles about COVID-19 optimal intervention modeling mainly
    focus on Nonpharmaceutical interventions (NPIs). Mostly these works
    understand the control strategy as the diminish of contact rates by
    reducing mobility or modulating parameters regarding the generation of new
    infections by linear controls \cite{Naraigh2020,Ullah2020},
    Lockdown-Quarantine \cite{Mandal2020},  shield immunity
    \cite{Weitz2020}. Libotte et. al. reports in \cite{Libotte2020} optimal
    vaccination strategies for COVID19.
\paragraph{Contribution and main objectives}
        Our manuscript is the first contribution modeling with optimal control
    of Lockdown-Vaccination strategies' effect to the best of our knowledge.
    Since health services' response will be limited by the vaccine stock
    and logistics, to implement in parallel NPIs is mandatory. We focus on
    formulating and studying via simulation the system Lockdown-Vaccination
    with recent and approved vaccine profile by the  Mexico Health council and
    developing optimal policies for the Lockdown release-input and Vaccine
    application doses.
\paragraph{Vaccine development}
        According to official Governmental communication in December, Mexico
    treated  \SI{36000000}{doses} Pfizer-Biotech, \SI{76000000}{doses} with
    Aztra-Seneca \SI{18000000}{doses} of Cansino-BIO. Other developments
    also are running the  third Phase, and with high probability,  in the
    third quarter of 2021, some of these developments will incorporate into
    Mexico's vaccine portfolio. Despite official agreements, each vaccine's
    delivery schedule is under uncertainty and-or subject to the approval
    of COFEPRIS.
\paragraph{Problem setup}
        The first accepted vaccine \textemdash Pfizer-BioNTech's BNT162b2
    \textemdash has an efficacy above \SI{90}{\percent}  and requires
    two doses to achieve immunity. The other mentioned developments have a very
    similar profile but require different logistic management and stock
    allocation.  Thus, we face designing a schedule of dose application subject
    to a given vaccine stock that will be applied in a given period. To this
    end, we formulate an optimal control problem that minimizes the burden of
    COVID-19 in DALYs [WhoDALY(2020)]. We also optimize the cost generated by
    the implementation of Vaccination in parallel with Lockdown.

\paragraph{Piecewise optimal policies}
        Comment about the solution of the underlying Optimal Control Problem
    \comment[id=SDIV]{David}
    One of the main features of our model is that we consider piecewise
    constant control policies instead of general measurable control policies
    (also called permanent controls) to minimize a cost functional. General
    control policies are difficult to implement since the authority has to make
    different choices every permanently. The optimal policies we find are
    constant in each interval of time and hence these policies are easier to
    implement.
        Optimal control problems with piecewise constant policies have been
    studied in different contexts. For instance, a solution method based on the
    gradient of the cost functional is studied in \cite{MR3223602}; convergence
    results of piecewise constant solutions to permanent solutions in
    linear-quadratic problems are given in \cite{MR3627992}; or, in
    \cite{CANTUNetAl}, a general numerical methodology to find piecewise
    constant solutions is proposed.
\paragraph{Papaer structure}